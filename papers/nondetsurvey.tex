\documentclass{beamer}
\usepackage{amsmath,epsfig,amssymb,stmaryrd,latexsym}
\hypersetup{pdfpagemode=FullScreen}
\setbeamertemplate{navigation symbols}{} 
\input{mymacros}
\input{prooftree}
\usepackage[latin1]{inputenc}
\usepackage[matrix,arrow]{xy}
\usetheme{Madrid}
\newcommand{\medsquig}[1]{\ \Rightarrow^{\hspace{-0.9em}\raisebox{0.3ex}{${#1}$}}\ }
\newcommand{\Proc}{\mathtt{Proc}}
\newcommand{\ttor}{\ \mathtt{or}\ }
\newcommand{\may}{\mathsf{may}}
\newcommand{\maymust}{\mathsf{may-must}}
\newcommand{\SBD}{\mathsf{SBD}}
\newcommand{\recapp}[3]{\mathtt{rec}^{{#1}}\ {#2}}
\newcommand{\red}[1]{\textcolor{red}{#1}}
\newcommand{\green}[1]{\textcolor{green}{#1}}
\newcommand{\yellow}[1]{\textcolor{yellow}{#1}}
\newcommand{\blue}[1]{\textcolor{blue}{#1}}
\newcommand{\ttchoosebot}{\mathtt{choose}^{\bot}\ }
\newcommand{\rensucc}[1]{{[+1]{#1}}}

\title[Nondeterminism survey]{Nondeterminism: many problems and (maybe) some solutions}
\author{Paul Blain Levy}\institute{University of Birmingham}

\begin{document}

\begin{frame}
\titlepage
\end{frame}

\begin{frame}\frametitle{Outline}
\tableofcontents  
\end{frame}

\section{Setting Up}

\subsection{Imperative and functional languages}

\begin{frame}\frametitle{Functional language: call-by-name FPC}

Types
\begin{displaymath}
  A \bnfgo A \rightarrow A \bnf \tsum_{i \in I} A_{i} \bnf \prod_{i \in I} A_{i} \bnf \ttX \bnf \ttrec \ttX.\ A  \betwixt \mbox{($I$ countable)}
\end{displaymath}
Terms 
\begin{eqnarray*}
  M & \bnfgo &  \ttx \bnf \tuple{i,M} \bnf \ttm M \ttas \{\tuple{i,\ttx}.M_{i}\}_{i \in I} \\
  & &   \bnf \lambda \ttx. M \bnf  MM \bnf Mi \bnf \lambda\{i.M_{i}\}_{i \in I} \\
  &  & \bnf \ttrec \ttx.\ M \bnf  \fold M \bnf \unfold M
\end{eqnarray*}

\pause

A \red{ground type} is $\tsum_{i \in I}1$ \\ \medskip
The \red{strategy types} are of the form $\prod \sum \prod \sum \prod \sum \ldots$ \medskip
% \begin{displaymath}
%   A \bnfgo \mu \ttX. \prod_{i \in I} \sum_{j \in J_{i}} A_{ij}
% \end{displaymath}

Cpo semantics: $\tsum$ denotes \red{lifted sum}

\end{frame}

\begin{frame}\frametitle{Big-step semantics (Cousot)}

  \begin{block}{Convergence}
    Terminal terms $T \bnfgo \lambda \ttx. M \bnf \lambda\{i.M_{i}\}_{i \in I} \bnf \tuple{i,M} $ \\
    \medskip Define convergence $M \Downarrow T$ inductively, e.g.
\begin{displaymath}
  \begin{prooftree}
    M \Downarrow \lambda \ttx. P \betwixt P[N/\ttx] \Downarrow T \justifies MN
    \Downarrow T
  \end{prooftree}
\end{displaymath}
\end{block}
\pause
\begin{block}{Divergence}
  Define divergence $M \Uparrow$ coinductively, e.g.
\begin{displaymath}
  \begin{prooftree}
    M \Downarrow \lambda \ttx. P \betwixt P[N/\ttx] \Uparrow \justifies M N
    \Uparrow
  \end{prooftree}
\end{displaymath}
\end{block}
  
\end{frame}

\begin{frame}\frametitle{Imperative Language}
  
  \begin{block}{Syntax}
 \begin{displaymath}
    M \bnfgo \ttprint c.\ M \bnf \ttx \bnf \ttrec \ttx.\ M \hspace{6em} c \in \alphabet
  \end{displaymath}
  Also allow countable mutual recursion.
\end{block}

\pause

\begin{block}{Small-step semantics}
\begin{eqnarray*}
  \ttprint c.\ M & \rightsquigup{c} & M \\
  \ttrec \ttx.\ M & \rightsquigarrow & M[\ttrec \ttx.\ M/\ttx]
\end{eqnarray*}
\end{block}
\pause

A program either 
\begin{itemize}
\item prints a finite string, then diverges
\item or prints an infinite string.
\end{itemize}


\end{frame}

\begin{frame}\frametitle{Medium step semantics}

  \begin{block}{Convergence}
    Define $M \medsquig{c} N$ inductively:
\begin{displaymath}
  \begin{prooftree}
    \strut \justifies \ttprint c.\ M \medsquig{c} M
  \end{prooftree} \betwixt \betwixt
  \begin{prooftree}
    M[\ttrec \ttx.\ M/\ttx] \medsquig{c} N \justifies \ttrec \ttx.\ M
    \medsquig{c} N
  \end{prooftree}
\end{displaymath}
\end{block}

\begin{block}{Divergence}
  Define $M \Uparrow$ coinductively:
\begin{displaymath}
  \begin{prooftree}
    M[\ttrec \ttx.\ M/\ttx] \Uparrow \justifies \ttrec \ttx.\ M \Uparrow
  \end{prooftree}
\end{displaymath}
\end{block}
\pause
We have
\begin{itemize}
\item $M \medsquig{c} N$ iff $M \rightsquigstar \rightsquigup{c} N$
\item $M \Uparrow$ iff $M \rightsquiginfty$
\end{itemize}
 
\end{frame}



\begin{frame}\frametitle{From imperative to functional}


\begin{displaymath}
\mbox{Define the strategy type}  \hspace{5em} \Proc \eqdef \sum_{c \in \alphabet} \Proc
\end{displaymath} 

$\ttx_{0},\ldots, \ttx_{n-1} \vdash M$ in the imperative language
\medskip

translates into $\ttx_{0}:\Proc, \ldots, \ttx_{n-1}:\Proc \vdash M:\Proc$ in the functional language. \\
\pause
\medskip
This translation preserves operational semantics. \\
\pause
\medskip
Since $\Proc$ denotes the domain $\alphabet^{*\omega}$, it preserves cpo semantics too. \\
\pause
\medskip  
We could also translate interactive input, using nontrivial $\tprod$.

\end{frame}

\subsection{Adding nondeterminism}

\begin{frame}\frametitle{Three kinds of nondeterminism}
  
We can add to the functional language various kinds of nondeterminism.

\begin{block}{\yellow{Binary erratic nondeterminism} $M \ttor M'$}
  
  Choose to go left (and evaluate $M$) or right (and evaluate $M'$)
\end{block}

\begin{block}{\yellow{Countable erratic nondeterminism} $\ttchoose n \in \nats.\ M_{n}$}
  
  Choose a number $n$, then evaluate $M_{n}$
\end{block}

\begin{block}{\yellow{Ambiguous nondeterminism} $M \amb M'$}
  
  Evaluate $M$ and $M'$ fairly, return whatever you get first.
  
  If $M$ returns after 1 step, and $M'$ returns after 10000 steps, could still
  return the latter.
  
  We require $M,M'$ to have $\tsum$ type.
\end{block}

\begin{block}{\yellow{Ground amb}}
  Provides $\amb$ at ground type only.
\end{block}


\end{frame}

\begin{frame}\frametitle{Expressiveness}

Define
\begin{eqnarray*}
 \bot & \eqdef & \ttrec \ttx.\ \ttx \\
  \ttchoosebot n\in \nats.\  M_{n} & \eqdef & \bot \ttor \ttchoose n \in \nats.\ M_{n}
\end{eqnarray*}

\begin{itemize}
\item Binary erratic nondeterminism can express $\ttchoosebot n \in \nats.\ M_{n}$.

  \begin{displaymath}
    \xymatrix{
 \bullet \ar[r] \ar[dr] & \bullet \ar[r] \ar[dr]  & \bullet \ar[r]\ar[dr] & \bullet \ar[r] \ar[dr] & \cdots \\
 & M_{0} & M_{1} & M_{2} & M_{3}
}
  \end{displaymath}
  
\item  Ground $\amb$ can express countable erratic nondeterminism, parallel-or and
  parallel-exists.
\end{itemize}
\end{frame}

% end of 7 frames

\section{Linear time equivalences}


 \begin{frame}\frametitle{Example Application}

%A customer service program is specified as follows. 
\pause

The program must not kill the customer.\\ \pause \red{safety property} \\
\medskip
\pause
The program must greet the customer.  \\ \pause \red{liveness property} \\
\medskip 
\pause
If the program insults the customer, it must apologize. \\ \pause \red{conditional liveness property} \\
\medskip
\pause
The program must stop insulting the customer. \\ \pause \red{infinite liveness property} \\


% No time limit on liveness properties.

 \end{frame} 

\subsection{May testing: imperative and functional}

\begin{frame}\frametitle{May testing}

The most basic equivalence on programs is \red{may testing}.

This asks: what are the things that we \red{may observe}?

Or equivalently, the things that we \red{definitely won't observe} (safety properties).
\pause

\begin{block}{Definition}
  $M \simeq_{\may} M'$ when, for every ground context $\context[\cdot]$,
\begin{displaymath}
  \context[M] \Downarrow n \Leftrightarrow \context[M'] \Downarrow n
\end{displaymath}
\end{block}


\begin{block}{Examples}
\begin{eqnarray*}
  M \ttor \bot & \simeq_{\may} & M \\
M \amb M' & \simeq_{\may} & M \ttor M'
\end{eqnarray*}
\end{block}

In the imperative language, closed terms $M$ and $M'$ are identified when they have the same finite traces.  

\end{frame}

\begin{frame}\frametitle{Rational continuity}
  
May testing has a continuity property that gives rise to \red{lower powerdomain} semantics.

\begin{block}{Definition of $\recapp{n}{\ttx.\
 M}{N}$}
\begin{eqnarray*}
  \recapp{0}{\ttx.\ M}{N} & \eqdef & \bot \\
  \recapp{n+1}{\ttx.\ M}{N} & \eqdef & M[\recapp{n}{\ttx.\ M}{N}/\ttx]
\end{eqnarray*}
\end{block}

\begin{block}{Theorem}
  If $\context[\ttrec \ttx.\ M]$ can print ``hello'', then there exists $n \in \nats$ such that
\begin{math}
  \context[\recapp{n}{\ttx.\ M}{N}] \mbox{ can print ``hello''.}
\end{math}
\end{block}

Cpo semantics for may testing is typically fully definable and fully abstract.

\end{frame}

\begin{frame}\frametitle{Translation doesn't preserve may-testing}
  
% link back to translation



\begin{eqnarray*}
\mbox{\red{Imperative}} \hspace{4em}  a. (b.\bot \ttor c.\bot) & \blue{\simeq_{\may}} &  a.b.\bot \ttor a.c. \bot \\
\mbox{\red{Functional}} \hspace{2em}  \tuple{a,\tuple{b,\bot} \ttor \tuple{c, \bot}} & &   \tuple{a,\tuple{b,\bot}} \ttor \tuple{a, \tuple{c,\bot}}
\end{eqnarray*}

These can be distinguished by the context

{\small \begin{displaymath}
  \ttm [\cdot] \ttas \left\{
    \begin{array}{ll}
\tuple{a,\ttx}. & 
 \ttm \ttx \ttas \left\{
   \begin{array}{ll}
  \tuple{b,\ttx}. & 
 \ttm \ttx \ttas \left \{
   \begin{array}{ll}
\tuple{c,\ttx}. & \tttrue \\
\tuple{\not= c, \ttx}. & \bot
   \end{array} \right. \\
  \tuple{\not= b,\ttx}. & \bot
   \end{array} \right. \\
\tuple{\not= a, \ttx}. & \bot
    \end{array} \right.
\end{displaymath} }

To rectify this, we need an \red{affine} target language (like Winskel's Affine HOPLA).

\end{frame}

\subsection{Infinite traces: imperative}

\begin{frame}\frametitle{Imperative language: infinite traces}

For a closed term $M$, its set of behaviours $[M] \in \pset(\alphabet^{*\infty})$ \\
\medskip \pause
The kernel of $[\,]$ is called \red{infinite trace equivalence}. \\
\medskip \pause
Probably the most obvious equivalence to consider. \\

\medskip \pause
Can recognize all the properties of our customer service program. \\

\medskip \pause
Can we give a denotational semantics that agrees with $[\,]$ on closed terms?




  
\end{frame}


\begin{frame}\frametitle{Infinite traces \\ What doesn't work (1): least fixpoint semantics}

In least fixpoint semantics, $\bot$ is the least fixpoint of the identity, so $\bot \leqslant M$.  \\
\medskip

Consider
\begin{eqnarray*}
  M & \eqdef & \bot \err \printa{\printb{\bot}} \\
  M' & \eqdef & \bot \err \printa{\bot} \err \printa{\printb{\bot}}
\end{eqnarray*}
We have
\begin{eqnarray*}
  M = & \bot \err \bot \err \printa{\printb{\bot}} & \leqslant M' \\
 M = & \bot \err \printa{\printb{\bot}} \err \printa{\printb{\bot}} & \geqslant M'
\end{eqnarray*}
So $M=M'$, contradicting infinite trace equivalence.
  
\end{frame}

\begin{frame}\frametitle{Infinite traces \\ What doesn't work (2): well-pointed semantics}
  
In well-pointed semantics, a term in context $\Gamma$ denotes a \red{function} from a set of \red{environments}. \\
\medskip 
Linked to a \red{context lemma}: two terms that are equivalent in every \red{environment} are
    equivalent in every \red{program context}.   \\
\medskip
That is false for our language, in the case that $\alphabet = \{\tick\}$. 


\end{frame}

\begin{frame}\frametitle{Infinite traces \\ Counterexample to context lemma}

Here are two terms with a free identifier $\ttx$.
\begin{eqnarray*}
      N  & = &  \blue{\ttchoosebot n \in \nats.\ \tick^{n}.\ \bot} \err \ttx  \\
  N' & = &  \blue{\ttchoosebot n \in \nats.\ \tick^{n}.\ \bot} \err \ttx \red{\err \printtick{\ttx}}
\end{eqnarray*}
\medskip
\center{\begin{tabular}{c||c|c}
& $\tick^{n}$, then diverge & $\tick^{\omega}$ \\ \hline
$N$ & yes  & iff $\ttx$ can  \\
$N'$ & yes   & iff $\ttx$ can \\ \hline
$\ttrec \ttx.\ N$ &   yes & \red{no} \\
$\ttrec \ttx.\ N'$ & yes & \red{yes}
\end{tabular}}

  
\end{frame}

\begin{frame}\frametitle{Infinite traces \\ What does work: game semantics}
  
\begin{eqnarray*}
      N  & = &  \blue{\ttchoosebot n \in \nats.\ \tick^{n}.\ \bot} \err \ttx  \\
  N' & = &  \blue{\ttchoosebot n \in \nats.\ \tick^{n}.\ \bot} \err \ttx \red{\err \printtick{\ttx}}
\end{eqnarray*}
\medskip
\begin{itemize}
\item Somehow we have to distinguish $N$ and $N'$.  
\item $N'$ can print $\tick$, then \red{force} (i.e.\ execute) $\ttx$.  
\item Make forcing \red{explicit} in the denotational semantics.
\end{itemize}

\end{frame}

\subsection{Seeing Beyond Divergence: imperative}

\begin{frame}\frametitle{Seeing Beyond Divergence (1)}


  \begin{block}{Definition of $[M]_{\catn}$}
\begin{itemize}
\item the set of finite traces of $M$, together with extensions of divergences
\item the set of extensions of divergences of $M$
\item the set of infinite traces of $M$, together with extensions of divergences
\end{itemize}
\end{block}
This semantics is \red{divergence strict}. \\
\medskip \pause
To model recursion, we take the \red{greatest} fixpoint.  (Reverse ordering is the upper powerdomain.)




\end{frame}

\begin{frame}\frametitle{Seeing Beyond Divergence (2)}


  \begin{block}{Definition of $[M]_{\SBD}$} 
\begin{itemize}
\item the set of finite traces of $M$
\item the set of divergences of $M$
\item the set of infinite traces, together with limits of divergences (called
  ``$\omega$-divergences'')
\end{itemize}
\end{block}
\medskip \pause
To model recursion:
\begin{itemize}
\item first compute the greatest fixpoint wrt $[\,]_{\catn}$, giving an equivalence class for $[\,]_{\SBD}$
\item then compute the least fixpoint wrt $[\,]_{\SBD}$ within that class.
\end{itemize}
This is called the \red{reflected} fixpoint.
  
\end{frame}

\subsection{May and must testing: functional}

% end of 15 frames

\begin{frame}\frametitle{Functional language: may and must testing}

% In addition to safety properties, let us consider \red{liveness} properties, e.g.
% \begin{table}{l}
%   $M$ must return an even number. \\
%   $M$ must print either ``Hello'' or ``Good morning''.
% \end{table}

For a set $A \subseteq \nats$, we want to observe whether a program \red{must} return a value in $A$.
 
\medskip

This is a \red{liveness} property.

\begin{block}{Definition}
  For two terms $M,M'$, say $M \simeq_{\maymust} M'$ when for every ground
  context $\context[\cdot]$, we have
\begin{eqnarray*}
  \context[M] \Downarrow n & \Leftrightarrow & \context[M'] \Downarrow n \\
  \context[M] \Uparrow & \Leftrightarrow & \context[M'] \Uparrow
\end{eqnarray*}
\end{block}

The context lemma holds for (binary or countable) erratic nondeterminism under this equivalence. [Lassen]


\end{frame}



\begin{frame}\frametitle{Powerdomain semantics for may-must: binary nondeterminism}



For binary nondeterminism, we have a continuity property for must-testing.

\begin{block}{Theorem}
  For any $A \subseteq \nats$, if $\context[\ttrec \ttx.\ M]$ must return an
  element of $A$, then there exists $n$ such that

\begin{math}
  \context[\recapp{n}{\ttx.\ M}{N}] \mbox{ must return an element of $A$.}
\end{math}
\end{block}
\medskip
This leads to \red{convex powerdomain} semantics for $\simeq_{\maymust}$ [Plotkin]. \\
\medskip \pause
\red{Problem} The model contains undefinable elements even at first order, causing failure of full abstraction at second order.


\end{frame}

\begin{frame}\frametitle{May-must equivalence for countable nondeterminism \\ What doesn't work: continuity}

Continuous semantics cannot recognize divergence.

\begin{block}{Proof (Apt-Plotkin)}
  Define $A \eqdef  \tprod_{n \in \nats}\ttbool$ and define $\ttf:A \vdash M : A$ to be

\begin{displaymath}
 \lambda \left\{
    \begin{array}{ll}
0. & \ttchoose n>0.\ \ttf(n) \\
1. & \tttrue \\
n >1. & \ttf(n-1)
    \end{array} \right.
\end{displaymath}

and $\context[\cdot]$ to be $[\cdot]0$.  Then, up to may-must equivalence,
\begin{eqnarray*}
  \context[\recapp{k}{\ttf.\ M}{\bot}{}]{} & \mbox{ is } & \tttrue \ttor \bot \\
  \context[\ttrec \ttf.\ M] & \mbox{ is } & \tttrue 
\end{eqnarray*}
\end{block}

Plotkin et al developed a variant of the convex powerdomain for countable nondeterminism, using transfinite approximants.

  
\end{frame}

\begin{frame}\frametitle{Amb \\  What doesn't work (1): least fixpoint semantics}
  
How can we give denotational semantics for amb? \\
\medskip

Least fixpoint semantics doesn't work, even for ground amb. \\

\begin{displaymath}
    \tttrue \ttor \bot \leqslant  \tttrue \ttor \tttrue 
  =  \tttrue
\end{displaymath}
\begin{eqnarray*}
  \tttrue & = & \ttif (\ttfalse \amb \bot) \ttthen \bot \ttelse \tttrue \\
 & \leqslant & \ttif (\ttfalse \amb \tttrue) \ttthen \bot \ttelse \tttrue \\
 & = & \tttrue \ttor \bot
\end{eqnarray*}

So $\tttrue \ttor \bot = \tttrue$.  That's may-testing.

\end{frame}

\begin{frame}\frametitle{Amb \\ What doesn't work (2): well-pointed semantics}
  
[2007] Amb \red{breaks the context lemma}. \\
\medskip
Let $A$ be the strategy type $\prod_{1} \sum_{1} \prod_{1} \sum_{1} 1$.   \\
\medskip
A closed term of type $A$ gives (operationally) an element of $[A] \eqdef \pset ((\pset (1_{\bot})_{\bot})$. \\
\pause \medskip
There exist two terms $\ttx:A \vdash M,M':A$ giving (operationally) the same endofunction on $[A]$ \\
\medskip
and a ground context $\context[\cdot]$ such that
\begin{itemize}
\item $\context[\ttrec \ttx.\ M]$ may diverge
\item $\context[\ttrec \ttx.\ M']$ must converge.
\end{itemize}
\medskip
So no well-pointed semantics is possible.  

\end{frame}

\begin{frame}\frametitle{Context lemma for ground amb}

[2007] The context lemma holds in the presence of ground amb. \\
\medskip

This suggests that there could be a well-pointed semantics for ground amb.

\end{frame}

\section{Branching time equivalence: imperative and functional}

\begin{frame}\frametitle{Lower and convex bisimulation: imperative language}
  
Let $\catr$ be a binary relation on closed terms.  \\
\medskip
It is a \red{lower simulation} when $M \ \catr \ M'$ and $M \medsquig{c} N$ implies $\exists N'$ such that $M' \medsquig{c} N'$ and $N \ \catr \ N'$. \\
\medskip
It is a \red{lower bisimulation} when $\catr$ and $\catr^{\op}$ are lower simulations. \\
\medskip
It is a \red{convex bisimulation} when moreover $M\ \catr\ M'$ implies $M \Uparrow \Leftrightarrow M' \Uparrow$.  \\
\medskip
The greatest lower bisimulation is called \red{lower bisimilarity}.

\medskip \pause
Two terms are lower bisimilar 
\begin{itemize}
\pause\item iff they satisfy the same formulas in \red{Hennessy-Milner logic}
\pause\item iff there is a strategy for the \red{bisimilarity game} between them
\pause \item iff they have the same \red{anamorphic image}.
\end{itemize}
\end{frame}

\begin{frame}\frametitle{Bisimilarity \\ What doesn't work: least fixpoint semantics}
  
Once again
\begin{eqnarray*}
  M & \eqdef & \bot \err \printa{\printb{\bot}} \\
  M' & \eqdef & \bot \err \printa{\bot} \err \printa{\printb{\bot}}
\end{eqnarray*}
We have
\begin{eqnarray*}
  M = & \bot \err \bot \err \printa{\printb{\bot}} & \leqslant M' \\
 M = & \bot \err \printa{\printb{\bot}} \err \printa{\printb{\bot}} & \geqslant M'
\end{eqnarray*}
So $M=M'$, but they are not lower bisimilar.



% Here are two programs that are 
% \begin{itemize}
% \item mutually lower similar \green{(indeed mutually convex similar)} 
% \item but are not lower bisimilar \green{(nor upper bisimilar)}.
% \end{itemize}



% \begin{eqnarray*}
%   N & = & a.\ \bot \\
% & & \err a.\ b.\ c.\ \bot \\ \\
% N' & = & a.\ \bot \\
% & & \err a.\ b.\ c.\ \bot \\
%  & & \err a.\ (b.\ \bot \err b.\ c.\ \bot)
% \end{eqnarray*}

\medskip



\end{frame}

\begin{frame}\frametitle{Lower similarity \\ What doesn't work: continuity [Boudol, Abramsky]}

Let the alphabet be $\nats$, and include a renaming operator $[+1]$. \\
\medskip
Let \blue{$\ttx \vdash M$} be \blue{$\bot \err ( 0.\ \bot) \err \rensucc{\ttx}$} \\
\medskip
Let \blue{$\context[\cdot]$} be \blue{$(\ttchoosebot n \in \nats.\  0.\ \ttchoosebot m\leqslant n.\  m.\ \bot) \err ( 0.\ [\cdot]) $} \\
\medskip
Then. up to convex bisimilarity,
\begin{eqnarray*}
  \context[\recapp{k}{\ttx.\ M}{}]{} & \mbox{ is } & (\ttchoosebot n \in \nats.\  0.\ \ttchoosebot m\leqslant n.\  m.\ \bot) \\
  \context[\ttrec \ttx.\ M] & \mbox{ is } & (\ttchoosebot n \in \nats.\  0.\ \ttchoosebot m\leqslant n.\  m.\ \bot) \\
& & \red{\err ( 0.\ \ttchoosebot m \in \nats.\  m.\ \bot)} 
\end{eqnarray*}
The latter and the former are not related by lower similarity. \\
\medskip
But they are identified by any continuous semantics.
  
\end{frame}



\begin{frame}\frametitle{Denotational semantics of lower bisimilarity}
 Is there a well-pointed semantics of lower bisimilarity?

 \begin{block}{Abramsky's domain equation}
   Abramsky presented a ``domain equation for bisimulation''. \\
   \medskip 
   If $M,M'$ have no divergences then $\seman{M}=\seman{M'}$ iff $M,M'$ are
   lower bisimilar. \\
   \medskip
   But for general programs, that is not the case.
\end{block}

No well-pointed semantics of lower bisimilarity is known: what kind of fixpoint should we use to interpret recursion? 
\end{frame}

\begin{frame}\frametitle{Applicative bisimilarity [Abramsky]: functional language}

A binary relation $\catr$ on closed terms is a \red{lower applicative simulation} when $M\ \catr\ M':A$ implies
\begin{itemize}
\item \blue{(if $A = B \rightarrow C$)} for all closed $N:B$ we have $MN\ \catr\ M' N$
\item \blue{(if $A = \prod_{i \in I} B_{i}$)} for all $i \in I$ we have $M i \ \catr \ M' i$
\item \blue{(if $A = \tsum_{i \in I} A_{i}$)} if $M \Downarrow \tuple{i,N}$ then $\exists N'$ such that $M' \Downarrow \tuple{i,N'}$ and $N \ \catr \ N'$.
\end{itemize}
\medskip
Lower and convex bisimulation are as before. \\
\medskip
The (imperative $\rightarrow$ functional) translation preserves and reflects lower and convex bisimilarity.
  
\end{frame}

\begin{frame}\frametitle{Properties of applicative bisimilarity}

Lower applicative bisimilarity is a congruence, by \red{Howe's method}.\\
\medskip

Convex applicative bisimilarity is a congruence in the presence of erratic nondeterminism, and [2007] of ground amb. \\

\medskip
But not in the presence of general amb (previous example). \\

\medskip  
In the nondeterministic setting, it is finer than may-must equivalence, e.g.
Boudol-Abramsky example.
\begin{displaymath}
  \begin{array}{l}
  \ttchoosebot n \in \nats.\ \tuple{i, \ttchoosebot m\leqslant n.\ m} \simeq_{\maymust}  \\
\ttchoosebot n \in \nats.\ \tuple{i, \ttchoosebot m\leqslant n.\ m} \red{\err
 \tuple{i, \ttchoosebot m \in \nats.\ m}}
\end{array}
\end{displaymath}

\end{frame}

\section{Lines of attack}


\begin{frame}\frametitle{Howe's method}

Howe's method, showing that applicative bisimilarity is a congruence, is \blue{elegant} but \textcolor{orange}{mysterious}. \\
\medskip
It assumes finitary syntax, but has recently been adapted for infinitary syntax. \\

\medskip \pause

\blue{Can we get some understanding of this method?} \\

\medskip 

B\"{o}hm trees---also represented as \red{innocent well-bracketed strategies}, in the deterministic setting---abstract away from syntactic detail.   \\

\medskip
 
Congruence of applicative bisimilarity says that composition of innocent well-bracketed strategies preserves applicative bisimilarity.  This may (?) be easier to understand.
  
\end{frame}


\begin{frame}\frametitle{Moving up the type hierarchy}
  
To model lower applicative bisimilarity we have to say what functions are definable, as we move up the type hierarchy. \\
\medskip

This is similar to the quest for a model of sequential computation.  \\
\medskip
So far, we can characterize definable functions between strategy types.  (Cf. Kahn-Plotkin sequentiality) \\

\medskip
This may be good enough for the imperative language. \\
\medskip
But we cannot yet characterize definability at higher-order.   \\
\medskip
Is it computable at finite types?  (Cf. Loader)

\end{frame}

\begin{frame}\frametitle{Model of bisimilarity: nested simulation}

A \red{2-nested} lower simulation is a simulation contained in mutual similarity.  \\
\medskip
 
A \red{3-nested} lower simulation is a simulation contained in mutual 2-nested similarity.  And so through all countable ordinals. \\
\medskip

The intersection of $n$-nested similarity for $n < \omega_{1}$ is bisimilarity.

\end{frame}

\begin{frame}\frametitle{Model of bisimilarity: suggested semantics}


A type denotes an $\omega_{1}$ sequence of preorders $\catr_{n}$ where $\catr_{n}$ is contained in the symmetrization of $\catr_{m}$, for every $m < n$, and the intersection of $\catr_{n}$ over all $n < \omega_{1}$ is the discrete relation. \\

\medskip

$\catr_{n}$ represents $n$-nested simulation. \\

\medskip

Given an endofunction on such a set, we compute a fixpoint by
\begin{itemize}
\item taking the least fixpoint for $\catr_{0}$, giving an equivalence class for $\catr_{1}$
\item take the least fixpoint for $\catr_{1}$ within this class, giving an equivalence class for $\catr_{2}$
\item etc.
\end{itemize}
\medskip

\blue{Is this the correct fixpoint?}

\end{frame}

\begin{frame}\frametitle{Summary}

  \begin{itemize}
  \item \red{Infinite traces}: now well understood.
  \item Fully abstract model of \red{may-must} testing?
  \item \red{Any} model of may-must testing with \red{ground amb}?
  \item \red{General amb}: many basic operational questions.
 \item After amb comes \red{fair merge}.
\item \red{Lower bisimilarity}: signs of progress.
\item Afterwards comes \red{convex bisimilarity}.
  \item Affineness
  \item Dataflow and call-by-need
  \end{itemize}

\end{frame}


\end{document}








